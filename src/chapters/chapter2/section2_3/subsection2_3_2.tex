\subsection*{\lr{2.3.2} جایگزینی}
      معادل منطقی توجیه‌کنندهٔ جایگزینی یک فرمول به‌جای فرمول دیگری است.
      
      \begin{definition}[تعریف \lr{2.30}]
        اگر $A$ زیردرختی از فرمول $B$ باشد، آنگاه می‌گوییم $A$ یک \emph{زیرفرمول} از $B$ است. اگر $A$ دقیقاً با $B$ یکسان نباشد، آن را \emph{زیرفرمول درست} $B$ می‌نامیم.
      \end{definition}
      
      \begin{example}[مثال \lr{2.31}]
        شکل \lr{2.4} فرمول
        \[
        (p \to q) \;\leftrightarrow\; (\neg p \to \neg q)
        \]
        (فرمول سمت چپ شکل \lr{2.1}) و زیرفرمول‌های درست آن را نشان می‌دهد.
        اگر به‌صورت رشته نمایش داده شود، زیرفرمول‌های درست عبارتند از:
        \[
        p \to q,\quad \neg p \to \neg q,\quad \neg p,\quad \neg q,\quad p,\quad q.
        \]
      \end{example}
      
      \begin{definition}[تعریف \lr{2.32}]
        فرض کنید $A$ یک زیرفرمول از $B$ باشد و $A'$ هر فرمول دلخواهی باشد.
        $B\{A \leftarrow A'\}$ یعنی جایگزینی $A'$ به‌جای $A$ در $B$، فرمولی است که از $B$ به‌دست می‌آید وقتی همهٔ زیردرخت‌های متناظر با $A$ در $B$ با $A'$ جایگزین شوند.
      \end{definition}
      
      \begin{example}[مثال \lr{2.33}]
        بگذارید
        \[
        B = (p \to q) \leftrightarrow (\neg p \to \neg q),\quad
        A = p \to q,\quad
        A' = \neg p \lor q.
        \]
        آنگاه
        \[
        B\{A \leftarrow A'\}
        \;=\;
        (\neg p \lor q) \;\leftrightarrow\; (\neg q \to \neg p).
        \]
      \end{example}
      
      اگر در فرمول $B$، یک زیرفرمول $A$ با فرمولی که معادل منطقی $A$ است جایگزین شود، ارزش‌گذاری $B$ در هیچ تعبیر تغییر نمی‌کند.
      
      \begin{theorem}[قضیه \lr{2.34}]
        فرض کنید $A$ زیرفرمولی از $B$ باشد و $A'$ فرمولی باشد که $A \equiv A'$. آنگاه
        \[
        B \;\equiv\; B\{A \leftarrow A'\}.
        \]
      \end{theorem}
      
      \begin{proof}
        بگذارید $\mathscr{I}$ یک تعبیر دلخواه باشد. چون $A \equiv A'$، پس:
        \[
        v_{\mathscr{I}}(A) = v_{\mathscr{I}}(A').
        \]
        باید نشان دهیم:
        \[
        v_{\mathscr{I}}(B)
        =
        v_{\mathscr{I}}\bigl(B\{A \leftarrow A'\}\bigr).
        \]
        اثبات با استقرا بر عمق $d$ بالاترین وقوع زیردرخت $A$ در $B$ انجام می‌شود:
        
        \begin{itemize}
          \item \textbf{حالت پایه} ($d = 0$):  
            در این حالت، تنها یک وقوع از $A$ وجود دارد که همان خود $B$ است. پس:
            \[
            v_{\mathscr{I}}(B)
            = v_{\mathscr{I}}(A)
            = v_{\mathscr{I}}(A')
            = v_{\mathscr{I}}\bigl(B\{A \leftarrow A'\}\bigr).
            \]
          \item \textbf{گام استقرا} ($d > 0$):  
            در این حالت، $B$ یکی از دو صورت زیر است:
            \begin{enumerate}
              \item $B = \neg B_1$,
              \item $B = B_1\;\mathrm{op}\;B_2$ برای برخی فرمول‌های $B_1, B_2$ و یک عملگر بولی $\mathrm{op}$.
            \end{enumerate}
            در هر دو صورت، عمق $A$ در زیردرخت‌های $B_1$ و $B_2$ کمتر از $d$ است. بنابراین، بر اساس فرض استقرا:
            \[
            v_{\mathscr{I}}(B_1)
            = v_{\mathscr{I}}\bigl(B_1\{A \leftarrow A'\}\bigr),
            \quad
            v_{\mathscr{I}}(B_2)
            = v_{\mathscr{I}}\bigl(B_2\{A \leftarrow A'\}\bigr).
            \]
            پس بر اساس تعریف ارزش‌گذاری برای عملگرهای بولی:
            \[
            v_{\mathscr{I}}(B)
            = v_{\mathscr{I}}\bigl(B\{A \leftarrow A'\}\bigr).
            \]
        \end{itemize}
        در نتیجه، از آنجا که $\mathscr{I}$ دلخواه بود داریم:
        \[
        B \equiv B\{A \leftarrow A'\}.
        \]
      \end{proof}