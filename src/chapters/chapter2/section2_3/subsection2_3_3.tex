\subsection*{\lr{2.3.3}  فرمول‌های معادل منطقی}
      جایگزینی فرمول‌های معادل منطقی اغلب انجام می‌شود، برای مثال در ساده‌سازی فرمول‌ها، و آشنایی با معادل‌های پرکاربرد فهرست‌شده در این زیربخش ضروری است. اثبات آن‌ها از تعاریف ابتدایی به‌دست می‌آید و به‌عنوان تمرین باقی گذاشته شده است.
      
      \subsubsection*{جذب ثابت‌ها \lr{(Absorption of Constants)}}
      اجازه دهید نحو فرمول‌های بولی را طوری گسترش دهیم که دو گزارهٔ اتمی ثابت \(\text{true}\) و \(\text{false}\) را نیز شامل شود. (نمادهای دیگر: \(\top\) برای \(\text{true}\) و \(\bot\) برای \(\text{false}\).) معنای آن‌ها به صورت زیر تعریف می‌شود:
      \[
      \mathscr{I}(\text{true}) = T
      \quad\text{و}\quad
      \mathscr{I}(\text{false}) = F
      \]
      برای هر تعبیر \(\mathscr{I}\).  
      این نمادها را نباید با مقادیر صدق \(T\) و \(F\) که در تعریف ارزش‌گذاری به‌کار می‌روند، اشتباه گرفت. همچنین می‌توان \(\text{true}\) و \(\text{false}\) را به‌ترتیب به‌عنوان اختصار برای فرمول‌های زیر در نظر گرفت:
      \[
      p \lor \neg p
      \quad\text{و}\quad
      p \land \neg p.
      \]
      وقوع یک ثابت در فرمول ممکن است آن را چنان فرو بکاهد که عملگر دودویی بی‌نیاز شود یا حتی فرمول به یک ثابت تبدیل شود:
      \[
      \begin{aligned}
      A \lor \text{true} &\equiv \text{true}
      &\quad
      A \land \text{true} &\equiv A\\
      A \lor \text{false} &\equiv A
      &\quad
      A \land \text{false} &\equiv \text{false}\\
      A \to \text{true} &\equiv \text{true}
      &\quad
      \text{true} \to A &\equiv A\\
      A \to \text{false} &\equiv \neg A
      &\quad
      \text{false} \to A &\equiv \text{true}\\
      A \leftrightarrow \text{true} &\equiv A
      &\quad
      A \oplus \text{true} &\equiv \neg A\\
      A \leftrightarrow \text{false} &\equiv \neg A
      &\quad
      A \oplus \text{false} &\equiv A
      \end{aligned}
      \]
      
      \subsubsection*{عملوندهای همسان \lr{(Identical Operands)}}
      فروپاشی (collapse) همچنین وقتی رخ می‌دهد که هر دو عملوند یکسان باشند یا یکی نقیض دیگری:
      \[
      \begin{aligned}
      A &\equiv \neg\neg A,\\
      A &\equiv A \land A
      &\quad
      A &\equiv A \lor A,\\
      A \lor \neg A &\equiv \text{true}
      &\quad
      A \land \neg A &\equiv \text{false},\\
      A \to A &\equiv \text{true},\\
      A \leftrightarrow A &\equiv \text{true}
      &\quad
      A \oplus A &\equiv \text{false},\\
      \neg A &\equiv A \uparrow A
      &\quad
      \neg A &\equiv A \downarrow A.
      \end{aligned}
      \]
      
      \subsubsection*{جابجایی، پیمایش‌پذیری، و توزیع‌پذیری}
      اپراتورهای دودویی بولی—به‌جز «تضمین» (\(\to\))—هم جابجایی‌پذیرند و هم پیمایش‌پذیر:
      \[
      \begin{aligned}
      A \lor B &\equiv B \lor A
      &\quad
      A \land B &\equiv B \land A,\\
      A \leftrightarrow B &\equiv B \leftrightarrow A
      &\quad
      A \oplus B &\equiv B \oplus A,\\
      A \uparrow B &\equiv B \uparrow A
      &\quad
      A \downarrow B &\equiv B \downarrow A.
      \end{aligned}
      \]
      با ورود نقیض، جهت یک تضمین می‌تواند وارونه شود:
      \[
      A \to B \;\equiv\; \neg B \to \neg A.
      \]
      فرمول \(\neg B \to \neg A\) را \emph{مخالف‌مقدم} (contrapositive) \(\,A \to B\) می‌نامند.
      
      جمع‌گزاره (\(\lor\))، ضرب‌گزاره (\(\land\))، معادل (\(\leftrightarrow\)) و نامعادل (\(\oplus\)) پیمایش‌پذیرند:
      \[
      \begin{aligned}
      A \lor (B \lor C) &\equiv (A \lor B) \lor C
      &\quad
      A \land (B \land C) &\equiv (A \land B) \land C,\\
      A \leftrightarrow (B \leftrightarrow C) &\equiv (A \leftrightarrow B) \leftrightarrow C
      &\quad
      A \oplus (B \oplus C) &\equiv (A \oplus B) \oplus C.
      \end{aligned}
      \]
      ولی تضمین (\(\to\))، nor (\(\downarrow\)) و nand (\(\uparrow\)) پیمایش‌پذیر نیستند.
      
      همچنین، جمع‌گزاره و ضرب‌گزاره بر یکدیگر توزیع‌پذیرند:
      \[
      \begin{aligned}
      A \lor (B \land C) &\equiv (A \lor B) \land (A \lor C),\\
      A \land (B \lor C) &\equiv (A \land B) \lor (A \land C).
      \end{aligned}
      \]
      
      \subsubsection*{تعریف یک عملگر به‌واسطهٔ عملگر دیگر}
      در اثبات قضایا با استقرا ساختاری، گام استقرایی باید برای هر عملگر دودویی جداگانه انجام شود. این گام‌ها ساده‌تر می‌شوند اگر بتوان عملگرهای خاص را با جایگزینی زیرفرمول‌هایی حذف کرد و فقط از مجموعه‌ای از عملگرهای پایه بهره گرفت. مثلاً معادل (\(\leftrightarrow\)) را می‌توان با ترکیب تضمین و ضرب‌گزاره تعریف کرد.
      
      همچنین، برخی الگوریتم‌ها برای تبدیل فرمول به شکل نرمال، نیازمند حذف برخی عملگرها هستند. فهرست معادل‌هایی که برای این کار کاربرد دارند:
      \[
      \begin{aligned}
      A \leftrightarrow B &\equiv (A \to B) \land (B \to A)
      &\quad
      A \oplus B &\equiv \neg(A \to B) \lor \neg(B \to A),\\
      A \to B &\equiv \neg A \lor B
      &\quad
      A \to B &\equiv \neg(A \land \neg B),\\
      A \lor B &\equiv \neg(\neg A \land \neg B)
      &\quad
      A \land B &\equiv \neg(\neg A \lor \neg B),\\
      A \lor B &\equiv \neg A \to B
      &\quad
      A \land B &\equiv \neg(A \to \neg B).
      \end{aligned}
      \]
      تعریف ضرب‌گزاره برحسب جمع‌گزاره و نقیض و بالعکس را \emph{قضایای دمورگان} \lr{(De Morgan’s laws)} می‌نامند.