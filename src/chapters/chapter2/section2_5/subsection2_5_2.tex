\subsection*{\lr{2.5.2} ارضاپذیری یک مجموعه از فرمول‌ها}
    مفهوم \emph{ارضاپذیری} را می‌توان به \emph{مجموعه‌ای از فرمول‌ها} نیز تعمیم داد.
    
    \begin{definition}[تعریف \lr{2.42}]
    یک مجموعهٔ فرمول‌ها
    \[
    U = \{A_1, A_2, \dots\}
    \]
    هنگامی \emph{(به‌طور هم‌زمان) ارضاپذیر} است که تعبیری $\mathscr{I}$ وجود داشته باشد به‌گونه‌ای که برای همهٔ اندیس‌ها $i$ داشته باشیم:
    \[
    v_\mathscr{I}(A_i) = T.
    \]
    چنین تعبیری $\mathscr{I}$ را \emph{مدل} $U$ می‌نامیم.
    
    مجموعهٔ $U$ را \emph{ناتوان از ارضا} (unsatisfiable) می‌نامیم هرگاه برای \emph{هر} تعبیر $\mathscr{I}$، دست‌کم یک $i$ وجود داشته باشد که:
    \[
    v_\mathscr{I}(A_i) = F.
    \]
    \end{definition}
    
    \begin{example}[مثال \lr{2.43}]
    مجموعهٔ
    \[
    U_1 = \{p,\; \neg p \lor q,\; q \land r\}
    \]
    ارضاپذیر است، زیرا تعبیری که به هر سه اتم مقدار $T$ اختصاص دهد، همهٔ فرمول‌ها را ارضا می‌کند.
    
    اما مجموعهٔ
    \[
    U_2 = \{p,\; \neg p \lor q,\; \neg p\}
    \]
    ناتوان از ارضا است. هر فرمول از این مجموعه به‌تنهایی قابل ارضا است، ولی \emph{هم‌زمان با یکدیگر} ارضاپذیر نیستند.
    \end{example}
    
    اثبات قضایای ابتدایی زیر به‌عنوان تمرین باقی گذاشته شده‌اند:
    
    \begin{itemize}
      \item \textbf{قضیه \lr{2.44}}: اگر $U$ ارضاپذیر باشد، آنگاه برای هر $i$، مجموعهٔ
      \[
      U - \{A_i\}
      \]
      نیز ارضاپذیر است.
    
      \item \textbf{قضیه \lr{2.45}}: اگر $U$ ارضاپذیر باشد و $B$ فرمولی صادق همگانی (valid) باشد، آنگاه
      \[
      U \cup \{B\}
      \]
      نیز ارضاپذیر است.
    
      \item \textbf{قضیه \lr{2.46}}: اگر $U$ ناتوان از ارضا باشد، آنگاه برای هر فرمول $B$، مجموعهٔ
      \[
      U \cup \{B\}
      \]
      نیز ناتوان از ارضا است.
    
      \item \textbf{قضیه \lr{2.47}}: اگر $U$ ناتوان از ارضا باشد و برای بعضی $i$، فرمول $A_i$ صادق همگانی باشد، آنگاه
      \[
      U - \{A_i\}
      \]
      نیز ناتوان از ارضا خواهد بود.
    \end{itemize}