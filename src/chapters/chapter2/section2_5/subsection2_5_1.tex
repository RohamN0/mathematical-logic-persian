\subsection*{\lr{2.5.1} Decision Procedures in Propositional Logic}
    \begin{definition}[تعریف \lr{2.40}]
      فرض کنید $\mathscr{U} \subseteq \mathscr{F}$ مجموعه‌ای از فرمول‌ها باشد. الگوریتمی یک \emph{رویهٔ تصمیم} برای $\mathscr{U}$ است اگر برای هر فرمول دلخواه $A \in \mathscr{F}$ پایان یابد و پاسخ «بله» را بازگرداند اگر $A \in \mathscr{U}$ و پاسخ «خیر» را اگر $A \notin \mathscr{U}$.
      
      اگر $\mathscr{U}$ مجموعهٔ فرمول‌های قابل ارضا باشد، رویهٔ تصمیم برای $\mathscr{U}$ را \emph{رویهٔ تصمیم برای ارضاپذیری}\\ (\lr{decision procedure for satisfiability}) می‌نامیم\\ و به‌طور مشابه برای همگانی‌صادق‌بودن.
      
      با توجه به قضیهٔ \lr{2.39}، می‌توان از رویهٔ تصمیم برای ارضاپذیری به‌عنوان رویهٔ تصمیم برای همگانی‌صادق‌بودن استفاده کرد. برای تصمیم‌گیری دربارهٔ اینکه آیا فرمول $A$ همگانی‌صادق است یا نه، کافی است رویهٔ تصمیم برای ارضاپذیری را روی $\neg A$ اجرا کنیم. اگر گزارش دهد $\neg A$ قابل ارضا است، آنگاه $A$ همگانی‌صادق نیست؛ و اگر گزارش دهد $\neg A$ قابل ارضا نیست، آنگاه $A$ همگانی‌صادق است. چنین رویه‌ای را \emph{رویهٔ ابطال} \lr{(refutation procedure)} می‌نامند، زیرا به‌جای اثبات مستقیم اینکه فرمول همیشه درست است، تنها به جستجوی مثال نقض می‌پردازد که کارآمدی بیشتری دارد.
    \end{definition}
    
    وجود رویهٔ تصمیم برای ارضاپذیری در منطق گزاره‌ای بدیهی است، زیرا می‌توانیم برای هر فرمول یک جدول ارزش صدق بسازیم. جدول ارزش صدق در مثال \lr{2.21} نشان می‌دهد که $p \to q$ قابل ارضا ولی ناخودهمگانی‌صادق است؛ در مثال \lr{2.22} نشان داده شد که
    \[
    (p \to q) \;\leftrightarrow\; (\neg q \to \neg p)
    \]
    همگانی‌صادق است. مثال زیر یک فرمول ناتوان از ارضا را نمایش می‌دهد.
    
    \begin{example}[مثال \lr{2.41}]
      فرمول
      \[
      (p \lor q)\;\land\;\neg p\;\land\;\neg q
      \]
      ناتوان از ارضا است، زیرا در همهٔ سطرهای جدول ارزش صدق آن، مقدار \(F\) به‌دست می‌آید:
      
      \[
      \begin{array}{|c|c|c|c|c|c|}
      \hline
      p & q & p \lor q & \neg p & \neg q & (p \lor q)\land\neg p\land\neg q\\\hline
      $T$ & $T$ & $T$ & $F$ & $F$ & $F$\\
      $T$ & $F$ & $T$ & $F$ & $T$ & $F$\\
      $F$ & $T$ & $T$ & $T$ & $F$ & $F$\\
      $F$ & $F$ & $F$ & $T$ & $T$ & $F$\\\hline
      \end{array}
      \]
    \end{example}
    
    روش جدول ارزش صدق یک رویهٔ تصمیم بسیار ناکارآمد است؛ زیرا برای فرمولی با \(n\) اتم متمایز، باید آن را برای هر یک از \(2^n\) تعبیر ممکن ارزیابی کنیم. در فصول بعدی، رویه‌های تصمیم کارآمدتری برای ارضاپذیری ارائه خواهد شد، اگرچه بسیار بعید است رویهٔ تصمیمی وجود داشته باشد که برای همهٔ فرمول‌ها به‌طور کارآمد اجرا شود (به بخش \lr{6.7} مراجعه کنید).