\subsection*{\lr{2.5.4} نظریه‌ها}
    نتیجهٔ منطقی، مفهوم مرکزی در بنیان‌های ریاضیات است.  
    فرمول‌های منطقی همواره‌صادق مانند
    \[
    p \lor q \leftrightarrow q \lor p
    \]
    اهمیت چندانی در ریاضیات ندارند. آنچه جالب‌تر است، \textbf{فرض گرفتن مجموعه‌ای از فرمول‌ها به‌عنوان صادق} و سپس \textbf{بررسی پیامدهای منطقی آن‌ها} است.
    
    برای مثال، اقلیدس پنج اصل دربارهٔ هندسه را فرض گرفت و مجموعهٔ گسترده‌ای از نتایج منطقی را از آن‌ها نتیجه گرفت.  
    تعریف صوری یک «نظریهٔ ریاضی» به‌صورت زیر است:
    
    \begin{definition}[تعریف \lr{2.55}]
      بگذارید $\mathscr{T}$ یک مجموعه از فرمول‌ها باشد. $T$ \textbf{تحت نتیجه‌ٔ منطقی بسته است} هرگاه برای هر فرمول $A$، اگر
      \[
        \mathscr{T} \models A,
      \]
      آنگاه
      \[
      A \in \mathscr{T}.
      \]
      مجموعه‌ای از فرمول‌ها که تحت نتیجه‌ٔ منطقی بسته باشد، یک \textbf{نظریه (theory)} نامیده می‌شود.  
      فرمول‌های عضو $\mathscr{T}$، \textbf{قضیه‌های (theorems)} نظریه نامیده می‌شوند.
    \end{definition}
    
    نظریه‌ها با انتخاب مجموعه‌ای از فرمول‌ها به‌عنوان \textbf{اصول موضوع (axioms)} ساخته می‌شوند و سپس \textbf{نتایج منطقی} آن اصول به نظریه افزوده می‌شوند.
    
    \begin{definition}[تعریف \lr{2.56}]
      اگر $\mathscr{T}$ یک نظریه باشد، آنگاه گفته می‌شود $\mathscr{T}$ \textbf{قابل اصل‌گذاری (axiomatizable)} است هرگاه \textbf{مجموعه‌ای از فرمول‌ها} مانند $U$ وجود داشته باشد به‌طوری‌که
      \[
      \mathscr{T} = \{A \mid U \models A\}.
      \]
      مجموعهٔ $U$ اصول موضوع نظریهٔ $\mathscr{T}$ هستند.  
      اگر $U$ متناهی باشد، آنگاه گفته می‌شود که $\mathscr{T}$ \textbf{قابل اصل‌گذاری به‌صورت متناهی} است.
    \end{definition}
    
    برای مثال، نظریهٔ حساب (\lr{Arithmetic}) قابل اصل‌گذاری است:  
    مجموعه‌ای از اصول توسط پئانو (\lr{Peano}) ارائه شده که نتایج منطقی آن‌ها، قضیه‌های حساب هستند.  
    اما نظریهٔ حساب \textbf{قابل اصل‌گذاری به‌صورت متناهی نیست}، چرا که اصل استقرا (\lr{induction axiom}) به‌صورت \textbf{یک طرح اصل} است که برای هر خاصیت در حساب، نمونه‌ای دارد و نمی‌توان آن را با تنها یک اصل بیان کرد.