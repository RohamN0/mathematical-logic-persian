\subsection*{\lr{2.1.3} رفع ابهام در نمایش رشته‌ای}
   \subsubsection*{پرانتزها}
      
     ساده‌ترین روش برای جلوگیری از ابهام، استفاده از پرانتزها برای حفظ ساختار درخت هنگام تولید رشته است.
      
     \paragraph{الگوریتم \lr{2.7} (نمایش فرمول به‌صورت رشته با پرانتز)} \hfill \\
     \textbf{ورودی:} یک فرمول $A$ از منطق گزاره‌ای \\
     \textbf{خروجی:} نمایش رشته‌ای از $A$ با استفاده از پرانتزها
      
     رویه‌ی بازگشتی زیر را فراخوانی کن: \lr{Inorder(A)}
      
     \begin{latin}
      \begin{verbatim}
      Inorder(F):
          if F is a leaf:
              print its label
              return
          Let F1 and F2 be the left and right subtrees of F
          print '('
          Inorder(F1)
          print the label of the root of F
          Inorder(F2)
          print ')'
      \end{verbatim}
     \end{latin}
   
     اگر ریشه‌ی $F$ با نماد $\neg$ برچسب‌گذاری شده باشد، زیر‌درخت چپ نادیده گرفته می‌شود و مرحله‌ی \lr{Inorder(F1)} انجام نمی‌گیرد.
   
     در این حالت، دو فرمول موجود در شکل \lr{2.1} به دو رشته‌ی متفاوت نگاشته می‌شوند و دیگر ابهامی وجود ندارد:
     \[
     ((p \rightarrow q) \leftrightarrow ((\neg q) \rightarrow (\neg p)))
     \]
     \[
     (p \rightarrow (q \leftrightarrow (\neg (p \rightarrow (\neg q)))))
     \]
   
     مشکل استفاده از پرانتزها آن است که فرمول‌ها را طولانی کرده و خواندن و نوشتن آن‌ها را دشوار می‌سازد.
   
   \subsubsection*{اولویت}
     
     روش دوم برای رفع ابهام در فرمول‌های مبهم، تعریف \textbf{اولویت} (precedence) و \\ \textbf{جهت انجمنی} (associativity) بین عملگرها است، همان‌طور که در حساب معمول انجام می‌شود. برای مثال، عبارت \\ \lr{a * b * c + d * e} بلافاصله به‌صورت \[
     (((a * b) * c) + (d * e))
     \] تفسیر می‌شود.
     
     در مورد فرمول‌های منطقی، ترتیب اولویت از بالا به پایین به‌صورت زیر است:
     
     \[ \\
     \neg \\
     \]
     \[
     \land, \uparrow \\
     \]
     \[
     \lor, \downarrow \\
     \]
     \[
     \rightarrow \\
     \]
     \[
     \leftrightarrow, \oplus
     \]
     عملگرها به‌طور پیش‌فرض \textbf{از راست به چپ} گروه‌بندی می‌شوند (جهت انجاسی راست‌گرا)، یعنی فرمول $a \lor b \lor c$ به‌صورت $a \lor (b \lor c)$ تفسیر می‌شود.
     
     پرانتزها تنها زمانی استفاده می‌شوند که بخواهیم ترتیبی متفاوت از قواعد اولویت و جهت پیش‌فرض را نشان دهیم؛ مانند حساب معمول که در آن \lr{a * (b + c)} به معنای انجام جمع پیش از ضرب است.
     
     با حداقل استفاده از پرانتز، دو فرمول مثال در شکل \lr{2.1} را می‌توان به‌صورت زیر نوشت:
     
     \[
     p \rightarrow q \leftrightarrow \neg q \rightarrow \neg p
     \]
     \[
     p \rightarrow (q \leftrightarrow \neg (p \rightarrow \neg q))
     \]
     
     برای وضوح بیشتر، همیشه می‌توان از پرانتزهای اضافی استفاده کرد؛ مثلاً:
     \[
     (p \lor q) \land (q \lor r)
     \]
     
     عملگرهای بولی $\land$, $\lor$, $\leftrightarrow$, $\oplus$ انجاسی هستند، بنابراین در فرمول‌هایی که شامل تکرار این عملگرها هستند، اغلب پرانتزها حذف می‌شوند؛ مثلاً:
     \[
     p \lor q \lor r \lor s
     \]
     
     اما توجه داشته باشید که عملگرهای $\rightarrow$, $\downarrow$, $\uparrow$ \textbf{غیرانجاسی} هستند، بنابراین استفاده از پرانتز برای جلوگیری از ابهام الزامی است.
     
     اگرچه فرض بر این است که $\rightarrow$ راست‌گرا است و فرمول $p \rightarrow q \rightarrow r$ را به‌صورت $p \rightarrow (q \rightarrow r)$ تفسیر می‌کنیم، اما معمولاً برای وضوح بیشتر، آن را همراه با پرانتز می‌نویسیم:
     \[
     (p \rightarrow q) \rightarrow r
     \]
     
   \subsubsection*{نمادگذاری لهستانی \lr{(Polish Notation)}}
 
    اگر نمایش رشته‌ای یک فرمول با استفاده از پیمایش \textbf{پیش‌گرد} \lr{(preorder traversal)} درخت آن انجام شود، دیگر ابهامی نخواهیم داشت.
 
    \paragraph{الگوریتم \lr{2.8} (نمایش فرمول به‌صورت رشته در نماد لهستانی)} \hfill \\
    \textbf{ورودی:} یک فرمول $A$ از منطق گزاره‌ای \\
    \textbf{خروجی:} نمایش رشته‌ای از $A$ در نمادگذاری لهستانی
 
    رویه‌ی بازگشتی زیر را فراخوانی کن: \lr{Preorder(A)}
 
    \begin{latin}
      \begin{verbatim}
      Preorder(F):
          print the label of the root of F
          if F is a leaf:
              return
          Let F1 and F2 be the left and right subtrees of F
          Preorder(F1)
          Preorder(F2)
      \end{verbatim}
    \end{latin}
 
    اگر ریشه‌ی $F$ با نماد $\neg$ برچسب‌گذاری شده باشد، زیر‌درخت چپ نادیده گرفته می‌شود و مرحله‌ی \lr{Preorder(F1)} انجام نمی‌گیرد.
 
    \begin{example}[مثال \lr{2.9}] رشته‌های مربوط به دو فرمول موجود در شکل \lr{2.1} در نمادگذاری لهستانی به‌صورت زیر هستند:
      \[
      \leftrightarrow\ \rightarrow\ p\ q\ \rightarrow\ \neg p\ \neg q
      \]
      \[
      \rightarrow\ p\ \leftrightarrow\ q\ \neg\ \rightarrow\ p\ \neg q
      \]
    و اکنون دیگر هیچ ابهامی وجود ندارد.
    \end{example}
    فرمول‌هایی که به این صورت نمایش داده می‌شوند، گفته می‌شود در \textbf{نمادگذاری لهستانی} هستند، که به افتخار گروهی از منطق‌دانان لهستانی به رهبری \lr{Jan Łukasiewicz} نام‌گذاری شده است.
 
    ما نمایش درون‌گرد (\lr{infix}) را خواناتر می‌دانیم، چراکه در حساب معمول نیز رایج است. بنابراین نمادگذاری لهستانی معمولاً تنها برای نمایش درونی عبارات حسابی و منطقی در رایانه‌ها استفاده می‌شود. مزیت اصلی آن این است که می‌توان فرمول را در همان ترتیبی که نمادها ظاهر می‌شوند با استفاده از یک \lr{stack} ارزیابی کرد.
 
    اگر فرمول اول را به‌صورت معکوس بازنویسی کنیم (که به آن \textbf{نمادگذاری لهستانی معکوس} یا \lr{RPN} نیز گفته می‌شود):
 
    \[
    q\ \neg\ p\ \neg\ \rightarrow\ q\ p\ \rightarrow\ \leftrightarrow
    \]
 
    می‌توان آن را مستقیماً به دنباله‌ای از دستورات اسمبلی زیر ترجمه کرد:
 
    \begin{latin}
      \begin{verbatim}
      Push q
      Negate
      Push p
      Negate
      Imply
      Push q
      Push p
      Imply
      Equiv
      \end{verbatim}
    \end{latin}
 
    در این روش، عملگرها روی عملوندهایی که در بالای پشته قرار دارند اعمال می‌شوند، سپس عملوندها از پشته خارج (pop) شده و نتیجه روی پشته قرار داده می‌شود.