\subsection*{\lr{2.1.6} \quad دستور زبان صوری برای فرمول ها}
  (این زیر‌بخش مستلزم آشنایی با دستور زبان‌های صوری است.)
  به‌جای تعریف فرمول‌ها به‌صورت درختی، می‌توان آن‌ها را به‌شکل رشته‌هایی تعریف کرد که از یک دستور زبان مستقل از متن (\lr{context-free grammar}) تولید می‌شوند.
  \begin{definition}[تعریف \lr{2.13}]
    فرمول‌های منطق گزاره‌ای از دستور زبانی مستقل از متن تولید می‌شوند که نمادهای پایانی (\lr{terminal symbols}) آن به‌صورت زیر تعریف شده‌اند:
    \begin{itemize}
      \item مجموعه‌ای نامتناهی از نمادها به‌نام گزاره‌های اتمی \lr{$\mathcal{P}$}
      \item عملگرهای بولی مطابق با تعریف \lr{2.1}
    \end{itemize}
    قواعد تولید (\lr{production rules}) این دستور زبان به‌صورت زیر هستند:
    \begin{align*}
      \text{fml} &::= p \qquad \text{برای هر } p \in \mathcal{P} \\
      \text{fml} &::= \neg \text{ fml} \\
      \text{fml} &::= \text{fml op fml} \\
      \text{op} &::= \lor \mid \land \mid \rightarrow \mid \leftrightarrow \mid \oplus \mid \uparrow \mid \downarrow
    \end{align*}
    یک فرمول، رشته‌ای است که می‌توان آن را از نماد غیرپایانی \lr{fml} استخراج کرد. مجموعه‌ی تمام فرمول‌هایی که از این دستور زبان به‌دست می‌آیند با نماد $\mathscr{F}$ نمایش داده می‌شود.
    فرآیند تولید (اشتقاق) رشته‌ها در یک دستور زبان صوری را می‌توان با درخت‌های اشتقاق \\ (\lr{derivation trees}) نمایش داد . رشته‌ی نهایی را می‌توان از طریق خواندن برگ‌ها از چپ به راست به‌دست آورد.
  \end{definition}
  \begin{example}[مثال \lr{2.14}]
    در ادامه، فرآیند اشتقاق فرمول زیر در منطق گزاره‌ای نشان داده شده است:
    \[
    p \rightarrow q \leftrightarrow \neg p \rightarrow \neg q
    \]
    \begin{figure}[ht]
    \centering
    \begin{latin}
    \resizebox{0.7\textwidth}{!}{%
    \begin{tikzpicture}[
      level distance=1.5cm,
      sibling distance=4cm,
      level 2/.style={sibling distance=2cm},
      level 3/.style={sibling distance=1cm},
      edge from parent path={(\tikzparentnode.south) -- (\tikzchildnode.north)}
    ]
    \node {fml}
      child { 
        node {fml}
        child { 
        	node {fml}
        	child {node {$p$}}
        }
        child {
        	node {op}
        	child {node {$\Rightarrow$}}
        }
        child { 
        	node {fml} 
        	child {node {$q$}}
        }
      }
      child { 
        node {op}
        child {node {$\Leftrightarrow$}}
        child { 
          node {fml}
          child { 
            node {fml}
            child {node {$\neg$}}
            child {node {fml}
              child {node {$p$}}
            }
          }
          child {
            node {op}
            child {node {$\Rightarrow$}}
          }
          child { 
            node {fml}
            child {node {$\neg$}}
            child {node {fml}
              child {node {$q$}}
            }
          }
        }
      };
    \end{tikzpicture}
    }
    \end{latin}
    \renewcommand{\thefigure}{\lr{2.2}}
    \caption{اشتقاق درخت برای $p \rightarrow q \leftrightarrow \neg p \rightarrow \neg q$ }
    \end{figure}
    درخت مربوطه در شکل \lr{2.2} آمده است. گام‌های اشتقاق به‌صورت زیر هستند:
    \begin{align*}
      1.\quad & \text{fml} \\
      2.\quad & \text{fml op fml} \\
      3.\quad & \text{fml} \leftrightarrow \text{fml} \\
      4.\quad & \text{fml} \rightarrow \text{fml} \leftrightarrow \text{fml} \\
      5.\quad & p \rightarrow \text{fml} \leftrightarrow \text{fml} \\
      6.\quad & p \rightarrow q \leftrightarrow \text{fml} \\
      7.\quad & p \rightarrow q \leftrightarrow \text{fml} \rightarrow \text{fml} \\
      8.\quad & p \rightarrow q \leftrightarrow \neg \text{fml} \rightarrow \text{fml} \\
      9.\quad & p \rightarrow q \leftrightarrow \neg p \rightarrow \text{fml} \\
      10.\quad & p \rightarrow q \leftrightarrow \neg p \rightarrow \neg \text{fml} \\
      11.\quad & p \rightarrow q \leftrightarrow \neg p \rightarrow \neg q
    \end{align*}
  \end{example}
  روش‌هایی که در بخش \lr{2.1.2} برای رفع ابهام معرفی شدند، در اینجا نیز قابل‌استفاده هستند.
  همچنین می‌توان دستور زبان را طوری بازنویسی کرد که پرانتزها را به‌صورت درونی دربر گیرد:
  \begin{align*}
    \text{fml} &::= (\neg \text{fml}) \\
    \text{fml} &::= (\text{fml op fml})
  \end{align*}
  و سپس با تعریف قواعد اولویت، می‌توان تعداد پرانتزها را کاهش داد.
  
  \begin{figure}[ht]
  \centering
    \begin{RTL}
      \begin{tabbing}
      \hspace{6.5cm} \= \kill
        $v_{\mathscr{I}}(A) = \mathscr{I}_A(A)$ \> اگر $A$ یک اتم باشد \\
        $v_{\mathscr{I}}(\neg A) = T$ \> اگر $v_{\mathscr{I}}(A) = F$ \\
        $v_{\mathscr{I}}(\neg A) = F$ \> اگر $v_{\mathscr{I}}(A) = T$ \\
        $v_{\mathscr{I}}(A_1 \lor A_2) = F$ \> اگر $v_{\mathscr{I}}(A_1) = F$ و $v_{\mathscr{I}}(A_2) = F$ \\
        $v_{\mathscr{I}}(A_1 \lor A_2) = T$ \> در غیر این صورت \\
        $v_{\mathscr{I}}(A_1 \land A_2) = T$ \> اگر $v_{\mathscr{I}}(A_1) = T$ و $v_{\mathscr{I}}(A_2) = T$ \\
        $v_{\mathscr{I}}(A_1 \land A_2) = F$ \> در غیر این صورت \\
        $v_{\mathscr{I}}(A_1 \rightarrow A_2) = F$ \> اگر $v_{\mathscr{I}}(A_1) = T$ و $v_{\mathscr{I}}(A_2) = F$ \\
        $v_{\mathscr{I}}(A_1 \rightarrow A_2) = T$ \> در غیر این صورت \\
        $v_{\mathscr{I}}(A_1 \uparrow A_2) = F$ \> اگر $v_{\mathscr{I}}(A_1) = T$ و $v_{\mathscr{I}}(A_2) = T$ \\
        $v_{\mathscr{I}}(A_1 \uparrow A_2) = T$ \> در غیر این صورت \\
        $v_{\mathscr{I}}(A_1 \downarrow A_2) = T$ \> اگر $v_{\mathscr{I}}(A_1) = F$ و $v_{\mathscr{I}}(A_2) = F$ \\
        $v_{\mathscr{I}}(A_1 \downarrow A_2) = F$ \> در غیر این صورت \\
        $v_{\mathscr{I}}(A_1 \leftrightarrow A_2) = T$ \> اگر $v_{\mathscr{I}}(A_1) = v_{\mathscr{I}}(A_2)$ \\
        $v_{\mathscr{I}}(A_1 \leftrightarrow A_2) = F$ \> اگر $v_{\mathscr{I}}(A_1) \ne v_{\mathscr{I}}(A_2)$ \\
        $v_{\mathscr{I}}(A_1 \oplus A_2) = T$ \> اگر $v_{\mathscr{I}}(A_1) \ne v_{\mathscr{I}}(A_2)$ \\
        $v_{\mathscr{I}}(A_1 \oplus A_2) = F$ \> اگر $v_{\mathscr{I}}(A_1) = v_{\mathscr{I}}(A_2)$ \\
      \end{tabbing}
    \end{RTL}
    \renewcommand{\thefigure}{\lr{2.3}}
    \caption{مقادیر منطقی فرمول‌ها}
  \end{figure}