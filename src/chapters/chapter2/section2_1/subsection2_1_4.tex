\subsection*{\lr{2.1.4} \quad استقرا ساختاری}
 
  اگر یک عبارت حسابی مانند \lr{a * b + b * c} را در نظر بگیریم، به‌راحتی می‌توان دید که این عبارت از دو جمله تشکیل شده که با عملگر جمع ترکیب شده‌اند، و هر جمله‌ی جمعی نیز از دو عامل ضرب تشکیل شده است. به‌طور مشابه، هر فرمول گزاره‌ای را می‌توان بر اساس عملگر سطح بالای آن طبقه‌بندی کرد.
 
  \begin{definition}[تعریف \lr{2.10}]
    اگر $A \in \mathscr{F}$ و $A$ یک اتم نباشد، عملگری که ریشه‌ی درخت فرمول $A$ را برچسب‌گذاری می‌کند، \textbf{عملگر اصلی} (\lr{principal operator}) فرمول $A$ نامیده می‌شود.
  \end{definition}
  \begin{example}[مثال \lr{2.11}]
    عملگر اصلی فرمول سمت چپ در شکل \lr{2.1}، عملگر $\leftrightarrow$ و در فرمول سمت راست، عملگر $\rightarrow$ است.\\
  \end{example}

  استقرا ساختاری روشی برای اثبات این است که یک خاصیت برای همه‌ی فرمول‌ها برقرار است. این شکل از استقرا، مشابه \textbf{استقرای عددی} آشنایی است که برای اثبات خاصیت برای همه‌ی اعداد طبیعی به‌کار می‌رود (بخش \lr{A.6} را ببینید). در استقرای عددی:

  \begin{itemize}
    \item \textbf{گام پایه:} اثبات خاصیت برای صفر است، و
    \item \textbf{گام استقرا:} فرض می‌کنیم خاصیت برای عدد دلخواه $n$ برقرار است و سپس اثبات می‌کنیم که برای $n+1$ نیز برقرار خواهد بود.
  \end{itemize}

  بر اساس تعریف \lr{2.10}، یک فرمول یا:
  \begin{itemize}
    \item یک برگ با برچسب اتم است،
    \item یا یک درخت با عملگر اصلی و یک یا دو زیر‌درخت.
  \end{itemize}

  بنابراین، در استقرا ساختاری:
  \begin{itemize}
    \item \textbf{گام پایه:} اثبات خاصیت برای برگ‌ها (اتم‌ها) است.
    \item \textbf{گام استقرا:} اثبات خاصیت برای فرمولی است که از به‌کارگیری عملگر اصلی روی زیر‌فرمول‌ها به‌دست آمده، مشروط بر آنکه خاصیت برای آن زیر‌فرمول‌ها برقرار باشد.
  \end{itemize}

  \begin{theorem}[قضیه \lr{2.12} (استقرا ساختاری). ]
   برای اینکه نشان دهیم خاصیتی برای همه‌ی فرمول‌ها $A \in \mathscr{F}$ برقرار است، کافی است:
   
   \begin{enumerate}
     \item اثبات کنیم که خاصیت برای تمام اتم‌ها $p$ برقرار است.
     \item فرض کنیم خاصیت برای فرمولی مانند $A$ برقرار است و اثبات کنیم که خاصیت برای $\neg A$ نیز برقرار است.
     \item فرض کنیم خاصیت برای فرمول‌های $A_1$ و $A_2$ برقرار است، و اثبات کنیم که خاصیت برای $A_1 \mathbin{\operatorname{op}} A_2$ نیز برقرار است، برای هر یک از عملگرهای دوجمله‌ای.
   \end{enumerate}
  \end{theorem}

  \begin{proof}[اثبات. ]
    فرض کنیم $A$ یک فرمول دلخواه باشد و فرض کنیم که بندهای (۱)، (۲)، (۳) برای خاصیت مورد نظر اثبات شده‌اند. ما نشان می‌دهیم که خاصیت برای $A$ برقرار است، با استفاده از استقرای عددی بر حسب $n$، که ارتفاع درخت مربوط به $A$ است:
    \begin{itemize}
      \item اگر $n = 0$ باشد، درخت یک برگ است، بنابراین $A$ یک اتم $p$ است و خاصیت طبق بند (۱) برقرار است.
      \item اگر $n > 0$ باشد، زیر‌درخت‌های $A$ ارتفاعی برابر با $n - 1$ دارند، بنابراین طبق فرض استقرای عددی، خاصیت برای آن‌ها برقرار است. از آنجا که عملگر اصلی $A$ یا نقیض ($\neg$) است یا یکی از عملگرهای دوجمله‌ای، بنابراین طبق بند (۲) یا (۳)، خاصیت برای $A$ نیز برقرار است.
    \end{itemize}
  \end{proof}
  بعداً نشان خواهیم داد که تمام عملگرهای دوجمله‌ای را می‌توان با ترکیب نقیض و یکی از دو عملگر $\lor$ یا $\land$ تعریف کرد؛ بنابراین، برای اثبات خاصیت برای همه‌ی فرمول‌ها، کافی است از استقرا ساختاری با حالت پایه و تنها دو گام استقرا استفاده کنیم.