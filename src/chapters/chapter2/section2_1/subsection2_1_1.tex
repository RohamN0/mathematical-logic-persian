\subsection*{\lr{2.1.1} فرمول‌ها به‌صورت درختی}
  \begin{definition}[تعریف \lr{2.1}]
    نمادهایی که برای ساخت فرمول‌های منطق گزاره‌ای استفاده می‌شوند عبارت‌اند از:
    \begin{itemize}
      \item مجموعه‌ای نامحدود از نمادها \text{$\mathscr{F}$} به نام \textbf{گزاره‌های اتمی} (که به اختصار «اتم» نامیده می‌شوند). اتم‌ها با حروف کوچک از مجموعه‌ی \lr{\{p, q, r, ...\}} نمایش داده می‌شوند، که ممکن است دارای زیرنویس نیز باشند.
      \item عملگرهای بولی. نام‌ها و نمادهای مربوط به آن‌ها به شرح زیر است:
    \end{itemize}
  
    \begin{center}
      \begin{tabular}{|c|c|}
      \hline
        نام & نماد \\
        \hline
        نقیض & $\neg$ \\
        یای منطقی & $\lor$ \\
        هم‌بندی & $\land$ \\
        شرطی & $\rightarrow$ \\
        هم‌ارزی & $\leftrightarrow$ \\
        یا-انحصاری & $\oplus$ \\
        نور & $\downarrow$ \\
        نند & $\uparrow$ \\
      \hline
      \end{tabular}
    \end{center}
  
    عملگر $\neg$ یک‌جمله‌ای (unary) است و تنها یک عملوند می‌گیرد، در حالی که سایر عملگرها دوجمله‌ای (binary) هستند و دو عملوند می‌پذیرند.
  \end{definition}
  \begin{definition}[تعریف \lr{2.2}]
  یک فرمول در منطق گزاره‌ای، درختی است که به‌صورت بازگشتی تعریف می‌شود:
    \begin{itemize}
      \item یک \textbf{برگ} با برچسب یک گزاره‌ی اتمی، یک فرمول است.
      \item \textbf{گره‌ای} با برچسب $\neg$ و تنها یک فرزند که خود یک فرمول باشد، یک فرمول است.
      \item \textbf{گره‌ای} با برچسب یکی از عملگرهای دوجمله‌ای و دو فرزند که هر دو فرمول باشند، یک فرمول است.
    \end{itemize}
  \end{definition}
  \begin{example}[مثال \lr{2.3}]
    شکل \lr{2.1} دو فرمول مختلف را نمایش می‌دهد.
  \end{example}