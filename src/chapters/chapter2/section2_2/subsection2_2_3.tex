\subsection*{\lr{2.2.3} فهم عملگرهای بولی}
      خوانش طبیعی از عملگرهای بولی «نقیض» ($\neg$) و «و» ($\land$) با معناهای رسمی‌ای که در شکل \lr{۲.۳} تعریف شدند، منطبق است. عملگرهای «ناند» ($\uparrow$) و «نُر» ($\downarrow$) صرفاً نقیض‌های $\land$ و $\lor$ هستند. در این‌جا دربارهٔ عملگرهای «یا» ($\lor$)، «یاِحصری» ($\oplus$) و «التزام» ($\rightarrow$) که معناهای رسمی‌شان می‌تواند موجب سردرگمی شود، توضیح می‌دهیم.
      \subsubsection*{«یا»ی شمولی در برابر «یاِحصری»}
        عملگر «یا» ($\lor$) همان یاِ شمولی است و با «یاِحصری» ($\oplus$) متفاوت است. برای مثال، فرض کنید می‌گوییم:
        \begin{quote}
        ساعت هشت «به سینما می‌روم» یا «به تئاتر می‌روم».
        \end{quote}
        منظور از این گزاره عبارت است از «سینما» $\oplus$ «تئاتر»، چرا که در یک لحظه نمی‌توان هم‌زمان در هر دو مکان بود. این در حالی است که عملگر $\lor$ زمانی مقدار درستی (T) می‌گیرد که دست‌کم یکی از جملات صادق باشد:
        \begin{quote}
        از پاپ‌کورن یا از آب‌نبات می‌خواهید؟
        \end{quote}
        در این حالت می‌توان گفت: «پاپ‌کورن» $\lor$ «آب‌نبات»، زیرا ممکن است هم پاپ‌کورن و هم آب‌نبات را بخواهیم.  
        برای $\lor$ کافی است یکی از زیرجملات صادق باشد تا کلِ عبارت صادق شود؛ بنابراین عبارت نامأنوس زیر نیز صادق است، تنها به این دلیل که جملهٔ اول به تنهایی کافی است:
        \[
        \text{«زمین دورتر از خورشید است تا ونوس»} \lor \text{«$1+1=3$»}
        \]
      \paragraph{تفاوت وقتی هر دو زیرجمله صادق باشند}
        \begin{itemize}
          \item با $\lor$: اگر دو زیرجمله صادق باشند، یاِ شمولی همچنان صادق است.
          \item با $\oplus$: اگر دو زیرجمله صادق باشند، یاِحصری نادرست است (چرا که حصری بودن ایجاب می‌کند دقیقاً یکی صادق باشد).
        \end{itemize}
        اینگونه می‌توان تفکیک روشنی میان $\lor$ و $\oplus$ قائل شد.
      \subsubsection*{یا شمولی در برابر یا اختصاصی در زبان‌های برنامه‌نویسی}
      وقتی \lr{or} در زمینهٔ زبان‌های برنامه‌نویسی به کار می‌رود، معمولاً منظور همان یا شمولی است:
        \begin{latin}
        \begin{verbatim}
        if (index < min || index > max) /* There is an error */
        \end{verbatim}
        \end{latin}
        در این مثال، درستی یکی از دو زیرعبارت باعث اجرای دستورات بعدی می‌شود. عملگر \verb| || | در اصل یک عملگر بولی واقعی نیست، زیرا از «ارزیابی کوتاه‌مدت» \lr{(short-circuit evaluation)} استفاده می‌کند: اگر زیرعبارت اول درست باشد، زیرعبارت دوم اصلاً ارزیابی نمی‌شود، چرا که نتیجهٔ آن نمی‌تواند تصمیم به اجرای ادامهٔ دستورات را تغییر دهد.
        برای ارزیابی بولی واقعی می‌توان از عملگر \verb!|! استفاده کرد؛ این عملگر معمولاً هنگام کار با بردارهای بیتی به کار می‌رود:
        \begin{latin}
        \begin{verbatim}
        mask1 = 0xA0;
        mask2 = 0x0A;
        mask  = mask1 | mask2;
        \end{verbatim}
        \end{latin}
        یاِحصری ($\oplus$ در منطق) که در زبان‌های برنامه‌نویسی با نماد \verb|^| نمایش داده می‌شود، برای رمزگذاری و رمزگشایی در سیستم‌های تصحیح خطا و رمزنگاری کاربرد دارد. دلیل این کار این است که با استفادهٔ دوباره از آن می‌توان مقدار اصلی را بازیافت. فرض کنید داده‌ای را با یک کلید مخفی رمزگذاری کنیم:
        \begin{latin}
        \begin{verbatim}
        codedMessage = data ^ key;
        \end{verbatim}
        \end{latin}
        گیرندهٔ پیام نیز می‌تواند به این صورت آن را رمزگشایی نماید:
        \begin{latin}
        \begin{verbatim}
        clearMessage = codedMessage ^ key;
        \end{verbatim}
        \end{latin}
        با دقت در محاسبهٔ زیر می‌بینیم که مقدار اولیه بازیابی می‌شود:
        \begin{latin}
        \begin{verbatim}
        clearMessage == codedMessage ^ key
                     == (data ^ key) ^ key
                     == data ^ (key ^ key)
                     == data ^ false
                     == data
        \end{verbatim}
        \end{latin}
        
      \subsubsection*{التزام (Implication)}
        عملگر $p \rightarrow q$ «التزام مادی» نامیده می‌شود؛ $p$ مقدم و $q$ مؤخر نامیده می‌شود. التزام مادی ادعای علیّت نمی‌کند؛ یعنی نمی‌گوید که مقدم، باعث مؤخر شده یا حتی با آن مرتبط است. یک التزام مادی تنها بیان می‌کند که اگر مقدم درست باشد، مؤخر نیز باید درست باشد؛ بنابراین تنها زمانی که مقدم درست و مؤخر نادرست باشد می‌توان آن را نادرست یافت.
        برای مثال، به دو گزارهٔ زیر توجه کنید:
        \[
        \text{«زمین دورتر از خورشید است تا ونوس»} \rightarrow \text{«$1+1=3$»}
        \]
        این گزاره نادرست است، چرا که مقدم درست و مؤخر نادرست است. اما:
        \[
        \text{«زمین دورتر از خورشید است تا مریخ»} \rightarrow \text{«$1+1=3$»}
        \]
        این گزاره صادق است، زیرا نادرستی مقدم به تنهایی برای صادق بودن کل التزام کافی است.