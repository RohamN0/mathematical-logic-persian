\section*{\lr{2.4} مجموعه‌ای از عملگرهای بولی}

      از دوران ابتدایی مدرسه به ما آموخته‌اند که چهار عملگر پایه در حساب عبارت‌اند از: جمع، تفریق، ضرب و تقسیم. بعدها با عملگرهای اضافی‌تری مانند مدولو (مانده) و قدر مطلق آشنا می‌شویم. از سوی دیگر، از دید نظری می‌دانیم ضرب و تقسیم در حقیقت زائد هستند، زیرا می‌توان آن‌ها را برحسب جمع و تفریق تعریف کرد.
      
      در این بخش، به دو پرسش می‌پردازیم:
      \begin{enumerate}
        \item چه عملگرهای بولی وجود دارند؟
        \item چه مجموعه‌ای از این عملگرها «کافی» است، به این معنا که بتوان همهٔ عملگرهای دیگر را تنها با استفاده از عملگرهای آن مجموعه تعریف کرد؟
      \end{enumerate}

\subsection*{\lr{2.4.1} عملگرهای بولی یگانی و دودویی}
      از آنجایی که تنها دو مقدار بولی $T$ و $F$ وجود دارد، تعداد عملگرهای با $n$ ورودی برابر است با $2^{2^n}$، زیرا برای هر یک از $n$ گزارهٔ ورودی می‌توان یکی از دو مقدار $T$ یا $F$ را انتخاب کرد (در مجموع $2^n$ ترکیب ممکن) و برای هر یک از این ترکیب‌ها می‌توان مقدار خروجی را $T$ یا $F$ قرار داد. در اینجا خود را به عملگرهای تک‌جایی (یک‌جایی) و دو‌جایی (دو‌جایی) محدود می‌کنیم.
      \paragraph{عملگرهای تک‌جایی}
      جدول زیر چهار عملگر تک‌جایی ممکن $\circ_1,\dots,\circ_4$ را نشان می‌دهد. ستون اول مقدار ورودی $x$ را و ستون‌های بعدی مقدار $\circ_n(x)$ را می‌دهند:
      \[
      \begin{array}{c|cccc}
      x & \circ_1 & \circ_2 & \circ_3 & \circ_4\\\hline
      $T$ & $T$ & $T$ & $F$ & $F$\\
      $F$ & $T$ & $F$ & $T$ & $F$
      \end{array}
      \]
      \paragraph{عملگرهای دو‌جایی}
      شکل \lr{2.5} عملگرهای دو‌جایی $\circ_1,\dots,\circ_{16}$ را بر اساس مقادیر ورودی $(x_1,x_2)$ فهرست می‌کند:
      \begin{figure}[ht]
      \centering
      \[
      \begin{array}{cc|cccccccc}
      x_1 & x_2 & \circ_1 & \circ_2 & \circ_3 & \circ_4 & \circ_5 & \circ_6 & \circ_7 & \circ_8\\\hline
      $T$ & $T$ & $T$ & $T$ & $T$ & $T$ & $T$ & $T$ & $T$ & $T$\\
      $T$ & $F$ & $T$ & $T$ & $T$ & $T$ & $F$ & $F$ & $F$ & $F$\\
      $F$ & $T$ & $T$ & $T$ & $F$ & $F$ & $T$ & $T$ & $F$ & $F$\\
      $F$ & $F$ & $T$ & $F$ & $T$ & $F$ & $T$ & $F$ & $T$ & $F$
      \end{array}
      \]
      \[
      \begin{array}{cc|cccccccc}
      x_1 & x_2 & \circ_{9} & \circ_{10} & \circ_{11} & \circ_{12} & \circ_{13} & \circ_{14} & \circ_{15} & \circ_{16}\\\hline
      $T$ & $T$ & $F$ & $F$ & $F$ & $F$ & $F$ & $F$ & $F$ & $F$\\
      $T$ & $F$ & $T$ & $T$ & $T$ & $T$ & $F$ & $F$ & $F$ & $F$\\
      $F$ & $T$ & $T$ & $T$ & $F$ & $F$ & $T$ & $T$ & $F$ & $F$\\
      $F$ & $F$ & $T$ & $F$ & $T$ & $F$ & $T$ & $F$ & $T$ & $F$
      \end{array}
      \]
      \renewcommand{\thefigure}{\lr{2.5}}
      \caption{عملگر های بولی دو جایی}
      \end{figure}
      از چهار عملگر تک‌جایی، سه تای آن‌ها بدیهی هستند: $\circ_1$ و $\circ_4$ عملگرهای ثابتند (همیشه $T$ یا همیشه $F$) و $\circ_2$ عملگر هویت است که ورودی را بدون تغییر برمی‌گرداند. تنها عملگر غیربدیهی تک‌جایی $\circ_3$ است که نقیض (negation) را پیاده می‌کند.
      برای عملگرهای دو‌جایی ($2^{2^2}=16$ حالت) نیز چند عملگر بدیهی وجود دارد:
      \begin{itemize}
        \item $\circ_1$ و $\circ_{16}$ عملگرهای ثابت،
        \item $\circ_4$ و $\circ_6$ عملگرهای \emph{projection} (مقدار خروجی تنها به یکی از ورودی‌ها بستگی دارد)،
        \item $\circ_{11}$ و $\circ_{13}$ منفیِ همین عملگرهای projection هستند.
      \end{itemize}
      جدول زیر تطابق نمادهای $\circ_n$ را با عملگرهای شناخته‌شده در تعریف \lr{2.1} نشان می‌دهد. نمادهای ستون راست منفیِ نمادهای همان‌ردیف در ستون چپ هستند.
      \begin{center}
      \begin{tabular}{|c|l|c||c|l|c|}
      \hline
      $\circ_n$ & نام عملگر           & نماد              & $\circ_m$  & نام عملگر     & نماد         \\\hline
      $\circ_2$ & جمع‌گزاره           & $\lor$            & $\circ_{15}$ & nor           & $\downarrow$ \\\hline
      $\circ_8$ & ضرب‌گزاره           & $\land$           & $\circ_{9}$  & nand          & $\uparrow$   \\\hline
      $\circ_5$ & تضمین (implication) & $\to$             & $\circ_{10}$ & xor (نامعادل) & $\oplus$     \\\hline
      $\circ_7$ & معادل               & $\leftrightarrow$ & —           & —             & —            \\\hline
      \end{tabular}
      \end{center}
      عملگر $\circ_{12}$ منفیِ تضمین است و معمولاً استفاده نمی‌شود. «تضمین معکوس» $\circ_3$ در زبان‌های منطق برنامه‌نویسی (فصل ۱۱) کاربرد دارد؛ منفیِ آن ($\circ_{14}$) نیز معمولاً به‌کار نمی‌رود.
\subsection*{\lr{2.4.2} مجموعه‌های کافی از عملگرها}
    \begin{definition}[تعریف \lr{2.35}]
      یک عملگر دودویی $\circ$ از مجموعهٔ عملگرها $\{\circ_1,\dots,\circ_n\}$ \emph{تعریف می‌شود} اگر و تنها اگر معادل منطقی
      \[
      A_1 \;\circ\; A_2 \;\equiv\; A
      \]
      وجود داشته باشد، جایی که $A$ فرمولی است ساخته‌شده از وقوع‌های $A_1$ و $A_2$ با استفاده از عملگرهای $\{\circ_1,\dots,\circ_n\}$.  
      عملگر یک‌جایی $\neg$ نیز زمانی \emph{تعریف‌شده} محسوب می‌شود که معادل منطقی
      \[
      \neg A_1 \;\equiv\; A
      \]
      وجود داشته باشد، با این توضیح که $A$ از وقوع‌های $A_1$ و عملگرهای مجموعه ساخته شده است.
    \end{definition}
    \begin{theorem}[قضیه \lr{2.36}]
      می‌توان همهٔ عملگرهای بولی 
      \[
      \lor,\;\land,\;\to,\;\leftrightarrow,\;\oplus,\;\uparrow,\;\downarrow
      \]
      را تنها از طریق $\neg$ و یکی از عملگرهای $\lor$، $\land$، یا $\to$ تعریف کرد.
    \end{theorem}
    \begin{proof}
      اثبات با استفاده از معادل‌های منطقی فهرست‌شده در زیربخش \lr{2.3.3} به‌دست می‌آید.  
      دو عملگر nand ($\uparrow$) و nor ($\downarrow$) به‌ترتیب منفیِ ضرب‌گزاره و جمع‌گزاره هستند.  
      معادل ($\leftrightarrow$) را می‌توان از تضمین ($\to$) و ضرب‌گزاره ($\land$) تعریف کرد و نامعادل ($\oplus$) را نیز از همین عملگرها و نقیض تعریف نمود.  
      بنابراین تنها نیاز به $\to$، $\lor$، $\land$ داریم، ولی هر یک از این سه عملگر را نیز می‌توان با استفاده از دیگری‌ها و نقیض تعریف کرد (مطابق معادل‌های صفحهٔ ۲۶).
      
      جالب آن است که می‌توان همهٔ عملگرهای بولی را تنها از طریق nand \emph{یا} تنها از طریق nor تعریف کرد. برای مثال، از معادل
      \[
      \neg A \;\equiv\; A \uparrow A
      \]
      برای تعریف نقیض از nand استفاده می‌کنیم. آنگاه برای ضرب‌گزاره داریم:
      \[
      \begin{aligned}
      (A \uparrow B) \uparrow (A \uparrow B)
      &\;\equiv\;
      \neg\bigl((A \uparrow B)\land(A \uparrow B)\bigr)
      &&\text{(تعریف \(\uparrow\))}\\
      &\;\equiv\;
      \neg\bigl(A \uparrow B\bigr)
      &&\text{(ایدمپوتنسی \(X\land X\equiv X\))}\\
      &\;\equiv\;
      \neg\neg(A\land B)
      &&\text{(تعریف \(\uparrow\))}\\
      &\;\equiv\;
      A\land B
      &&\text{(دوگانه‌نقیض)}.
    \end{aligned}
    \]
    پس از تعریف نقیض و ضرب‌گزاره از nand، می‌توان همهٔ عملگرهای دیگر را تعریف کرد. به‌طور مشابه، nor نیز مجموعهٔ کافی‌ای از عملگرها را تشکیل می‌دهد.  
    در واقع، می‌توان ثابت کرد که فقط nand و nor این خاصیت را دارند.
    \end{proof}
    \begin{theorem}[قضیه \lr{2.37}]
      فرض کنید $\circ$ یک عملگر دودویی باشد که بتوان با آن نقیض و همهٔ عملگرهای دودویی دیگر را تعریف کرد. آنگاه $\circ$ یا nand است یا nor.
    \end{theorem}
    \paragraph{طرح اثبات (خلاصه).}
    \begin{enumerate}
      \item از آنجا که $\circ$ باید نقیض را تعریف کند، باید معادلی از شکل
        \[
        \neg A \;\equiv\; A \circ \cdots \circ A
        \]
        وجود داشته باشد.
      \item هر عملگر دودویی $\mathrm{op}$ باید معادلی از صورت
        \[
        A_1 \;\mathrm{op}\; A_2 \;\equiv\; B_1 \circ \cdots \circ B_n
        \]
        داشته باشد، جایی که هر $B_i$ یا $A_1$ یا $A_2$ است (با پرانتزگذاری مناسب).
      \item با در نظر گرفتن یک تعبیر $\mathscr{I}$ که $v_{\mathscr{I}}(A)=T$ و استفاده از استقراء روی تعداد وقوع‌های $\circ$، نشان می‌دهیم که
        \[
        v_{\mathscr{I}}(A_1\circ A_2)=F
        \quad\text{وقتی}\quad
        v_{\mathscr{I}}(A_1)=T,\;v_{\mathscr{I}}(A_2)=T.
        \]
        به‌طور مشابه برای $v_{\mathscr{I}}(A)=F$ نتیجه می‌شود $v_{\mathscr{I}}(A_1\circ A_2)=T$.
      \item بنابراین، آزادی عمل $\circ$ فقط در حالتی است که دو عملوند مقادیر متفاوت داشته باشند:
        \begin{center}
        \begin{tabular}{|c|c|c|}
        \hline
        $A_1$ & $A_2$ & $A_1 \circ A_2$ \\ \hline
        $T$ & $T$ & $F$ \\
        $T$ & $F$ & $T$ یا $F$ \\
        $F$ & $T$ & $T$ یا $F$ \\
        $F$ & $F$ & $T$ \\ \hline
        \end{tabular}
        \end{center}
        اگر $\circ$ برای دو سطر وسط مقدار $T$ را انتخاب کند، $\circ$ همان nand است، و اگر $F$ را انتخاب کند، همان nor خواهد بود.
      \item تنها حالت باقیمانده این است که $\circ$ برای این دو سطر مقادیر متفاوت بدهد. با استقراء نشان دهید که در این صورت تنها می‌توان projection یا negated projection ساخت:
        \[
        B_1 \circ \cdots \circ B_n \;\equiv\; \neg\cdots\neg B_i
        \]
        برای بعضی $i$ و صفر یا چند نقیض.
    \end{enumerate}