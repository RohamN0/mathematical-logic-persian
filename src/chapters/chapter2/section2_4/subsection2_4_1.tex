\subsection*{\lr{2.4.1} عملگرهای بولی یگانی و دودویی}
      از آنجایی که تنها دو مقدار بولی $T$ و $F$ وجود دارد، تعداد عملگرهای با $n$ ورودی برابر است با $2^{2^n}$، زیرا برای هر یک از $n$ گزارهٔ ورودی می‌توان یکی از دو مقدار $T$ یا $F$ را انتخاب کرد (در مجموع $2^n$ ترکیب ممکن) و برای هر یک از این ترکیب‌ها می‌توان مقدار خروجی را $T$ یا $F$ قرار داد. در اینجا خود را به عملگرهای تک‌جایی (یک‌جایی) و دو‌جایی (دو‌جایی) محدود می‌کنیم.
      \paragraph{عملگرهای تک‌جایی}
      جدول زیر چهار عملگر تک‌جایی ممکن $\circ_1,\dots,\circ_4$ را نشان می‌دهد. ستون اول مقدار ورودی $x$ را و ستون‌های بعدی مقدار $\circ_n(x)$ را می‌دهند:
      \[
      \begin{array}{c|cccc}
      x & \circ_1 & \circ_2 & \circ_3 & \circ_4\\\hline
      $T$ & $T$ & $T$ & $F$ & $F$\\
      $F$ & $T$ & $F$ & $T$ & $F$
      \end{array}
      \]
      \paragraph{عملگرهای دو‌جایی}
      شکل \lr{2.5} عملگرهای دو‌جایی $\circ_1,\dots,\circ_{16}$ را بر اساس مقادیر ورودی $(x_1,x_2)$ فهرست می‌کند:
      \begin{figure}[ht]
      \centering
      \[
      \begin{array}{cc|cccccccc}
      x_1 & x_2 & \circ_1 & \circ_2 & \circ_3 & \circ_4 & \circ_5 & \circ_6 & \circ_7 & \circ_8\\\hline
      $T$ & $T$ & $T$ & $T$ & $T$ & $T$ & $T$ & $T$ & $T$ & $T$\\
      $T$ & $F$ & $T$ & $T$ & $T$ & $T$ & $F$ & $F$ & $F$ & $F$\\
      $F$ & $T$ & $T$ & $T$ & $F$ & $F$ & $T$ & $T$ & $F$ & $F$\\
      $F$ & $F$ & $T$ & $F$ & $T$ & $F$ & $T$ & $F$ & $T$ & $F$
      \end{array}
      \]
      \[
      \begin{array}{cc|cccccccc}
      x_1 & x_2 & \circ_{9} & \circ_{10} & \circ_{11} & \circ_{12} & \circ_{13} & \circ_{14} & \circ_{15} & \circ_{16}\\\hline
      $T$ & $T$ & $F$ & $F$ & $F$ & $F$ & $F$ & $F$ & $F$ & $F$\\
      $T$ & $F$ & $T$ & $T$ & $T$ & $T$ & $F$ & $F$ & $F$ & $F$\\
      $F$ & $T$ & $T$ & $T$ & $F$ & $F$ & $T$ & $T$ & $F$ & $F$\\
      $F$ & $F$ & $T$ & $F$ & $T$ & $F$ & $T$ & $F$ & $T$ & $F$
      \end{array}
      \]
      \renewcommand{\thefigure}{\lr{2.5}}
      \caption{عملگر های بولی دو جایی}
      \end{figure}
      از چهار عملگر تک‌جایی، سه تای آن‌ها بدیهی هستند: $\circ_1$ و $\circ_4$ عملگرهای ثابتند (همیشه $T$ یا همیشه $F$) و $\circ_2$ عملگر هویت است که ورودی را بدون تغییر برمی‌گرداند. تنها عملگر غیربدیهی تک‌جایی $\circ_3$ است که نقیض (negation) را پیاده می‌کند.
      برای عملگرهای دو‌جایی ($2^{2^2}=16$ حالت) نیز چند عملگر بدیهی وجود دارد:
      \begin{itemize}
        \item $\circ_1$ و $\circ_{16}$ عملگرهای ثابت،
        \item $\circ_4$ و $\circ_6$ عملگرهای \emph{projection} (مقدار خروجی تنها به یکی از ورودی‌ها بستگی دارد)،
        \item $\circ_{11}$ و $\circ_{13}$ منفیِ همین عملگرهای projection هستند.
      \end{itemize}
      جدول زیر تطابق نمادهای $\circ_n$ را با عملگرهای شناخته‌شده در تعریف \lr{2.1} نشان می‌دهد. نمادهای ستون راست منفیِ نمادهای همان‌ردیف در ستون چپ هستند.
      \begin{center}
      \begin{tabular}{|c|l|c||c|l|c|}
      \hline
      $\circ_n$ & نام عملگر           & نماد              & $\circ_m$  & نام عملگر     & نماد         \\\hline
      $\circ_2$ & جمع‌گزاره           & $\lor$            & $\circ_{15}$ & nor           & $\downarrow$ \\\hline
      $\circ_8$ & ضرب‌گزاره           & $\land$           & $\circ_{9}$  & nand          & $\uparrow$   \\\hline
      $\circ_5$ & تضمین (implication) & $\to$             & $\circ_{10}$ & xor (نامعادل) & $\oplus$     \\\hline
      $\circ_7$ & معادل               & $\leftrightarrow$ & —           & —             & —            \\\hline
      \end{tabular}
      \end{center}
      عملگر $\circ_{12}$ منفیِ تضمین است و معمولاً استفاده نمی‌شود. «تضمین معکوس» $\circ_3$ در زبان‌های منطق برنامه‌نویسی (فصل ۱۱) کاربرد دارد؛ منفیِ آن ($\circ_{14}$) نیز معمولاً به‌کار نمی‌رود.