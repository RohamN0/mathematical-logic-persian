 \subsection*{\lr{2.6.3} پایان‌پذیری ساخت جدول معنایی}

    از آنجا که هر گام از الگوریتم یک فرمول را به یک یا دو فرمول ساده‌تر فرو می‌شکند، بدیهی است که ساخت جدول معنایی برای هر فرمول خاتمه می‌یابد؛ با این حال، اثبات این ادعا ارزشمند است.
    
    \begin{theorem}[قضیه \lr{2.66}]
      ساخت جدول معنایی برای هر فرمول $\phi$ پایان‌پذیر است. هنگامی که ساخت خاتمه می‌یابد، تمام برگ‌ها با «$\times$» یا «$\odot$» علامت‌گذاری شده‌اند.
    \end{theorem}
    
    \begin{proof}
      فرض کنیم در فرمول $\phi$، عملگرهای $\leftrightarrow$ و $\oplus$ ظاهر نشوند (تعمیم به این دو مورد به‌صورت تمرین باقی گذاشته می‌شود).  
      برای هر برگ علامت‌نگرفته $l$ که در مرحله‌ای از گسترش انتخاب می‌شود، بگذارید
      \begin{align*}
      b(l) &\text{ تعداد کل عملگرهای دودویی موجود در همهٔ فرمول‌های }U(l),\\
      n(l) &\text{ تعداد کل نقیض‌ها در }U(l)
      \end{align*}
      سپس وزن
      \[
      W(l) \;=\; 3\,b(l) \;+\; n(l)
      \]
      را تعریف می‌کنیم.  
      برای مثال اگر
      \[
      U(l) = \{\,p \lor q,\;\neg p \land \neg q\},
      \]
      آنگاه $b(l)=2$ و $n(l)=2$ و بنابراین
      \[
      W(l) = 3\cdot 2 + 2 = 8.
      \]

      هر گام از الگوریتم یا یک گرهٔ جدید $l'$ یا دو گرهٔ جدید $l',l''$ را به‌عنوان فرزند $l$ می‌افزاید. ادعا می‌کنیم که در هر حالت:
      \[
      W(l') < W(l)
      \quad\text{و اگر گرهٔ دوم وجود داشته باشد،}\quad
      W(l'') < W(l).
      \]

      برای نمونه، فرض کنید فرمولی از نوع $\alpha$ داشته باشیم:
      \[
      A = \neg\,(A_1 \lor A_2),
      \]
      و قاعدهٔ $\alpha$ را روی برگ $l$ اعمال کنیم تا برگ جدید $l'$ برچسب‌خورد:
      \[
      U(l') = \bigl(U(l)\setminus\{\neg(A_1\lor A_2)\}\bigr)
      \;\cup\;\{\neg A_1,\;\neg A_2\}.
      \]
      در این صورت یکی از عملگرهای دودویی (یعنی $\lor$) و یک نقیض (علامت نفی بیرونی) حذف می‌شوند و دو نقیض جدید (برای $A_1$ و $A_2$) افزوده می‌شود. بنابراین:
      \[
      W(l') = W(l) - \bigl(3\cdot 1 + 1\bigr) + 2
      = W(l) - 2 < W(l).
      \]
      به همین ترتیب برای هر قاعدهٔ $\alpha$ یا $\beta$، وزن کاهش می‌یابد. از آنجا که $W(l)$ عددی طبیعی است و در هر گام کاهش پیدا می‌کند، الگوریتم نمی‌تواند بی‌پایان ادامه یابد و نهایتاً به برگ‌هایی منتهی می‌شود که همهٔ آن‌ها علامت‌گذاری شده‌اند.
    \end{proof}
